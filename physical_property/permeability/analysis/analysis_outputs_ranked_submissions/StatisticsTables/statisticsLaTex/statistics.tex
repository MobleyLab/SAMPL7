\documentclass{article}
\usepackage[a4paper,margin=0.005in,tmargin=0.5in,lmargin=0.5in,rmargin=0.5in,landscape]{geometry}
\usepackage{booktabs}
\usepackage{longtable}
\pagenumbering{gobble}
\begin{document}
\begin{center}
\scriptsize
\begin{longtable}{|ccccccccc|}
\toprule
    ID &                                  name &               RMSE &                MAE &                   ME &              R$^2$ &                   m &              $\tau$ &                    ES \\
\midrule
\endhead
\midrule
\multicolumn{9}{r}{{Continued on next page}} \\
\midrule
\endfoot

\bottomrule
\endlastfoot
 SGA28 &                                GROVER &  0.30 [0.21, 0.39] &  0.23 [0.15, 0.32] &  -0.08 [-0.21, 0.05] &  0.25 [0.02, 0.59] &   0.11 [0.02, 0.21] &  0.25 [-0.12, 0.55] &  -0.00 [-0.00, -0.00] \\
 QWO91 &  permeability-prediction-Attentive FP &  0.55 [0.42, 0.66] &  0.47 [0.34, 0.59] &    0.35 [0.16, 0.54] &  0.06 [0.00, 0.28] &  0.26 [-0.13, 0.70] &  0.12 [-0.19, 0.41] &  -0.00 [-0.00, -0.00] \\
\end{longtable}
\end{center}

Notes

- RMSE: Root mean square error

- MAE: Mean absolute error

- ME: Mean error

- R2: R-squared, square of Pearson correlation coefficient

- m: slope of the line fit to predicted vs experimental logP values

- $\tau$:  Kendall rank correlation coefficient

- ES: error slope calculated from the QQ Plots of model uncertainty predictions

- Mean and 95\% confidence intervals of RMSE, MAE, ME, R2, and m were calculated by bootstrapping with 10000 samples.

- 95\% confidence intervals of ES were calculated by bootstrapping with 1000 samples.\end{document}
